\documentclass{article}
\usepackage{graphicx} % Required for inserting images
\usepackage{algpseudocode}
\usepackage{algorithm}

\title{Introduction To Algorithmns and Data Structures}
\author{Finlay Ross-Davie }
\date{January 2024}


\begin{document}


\maketitle
\begin{abstract}
    Semester 2 Notes
\end{abstract}

\section{Interval Scheduling}

An algorithm is \textbf{greedy} if it builds up a solution in small steps, choosing a decision at each step myopically to optimize some underlying criterion \newline

In the interval scheduling problem, we have a set of requests ${1,2,\cdots,n}$ the $i^th$ request corresponds to an interval of time starting at $s(i)$ and finishing at $f(i)$. A subset of the requests is \textbf{compatible} if no two of them overlap in time, and the goal of this problem is to accept as large a compatible subset as possible. Compatible sets of maximum size will be called \textbf{optimal}. \newline

Three non-optimal greedy solutions to this problem are:
\begin{itemize}
    \item Always, select the available request with the minimal start time $s(i)$
    \item Begin by accepting the request which requires the smallest interval of time ie $f(i) - s(i)$ is as small as possible
\end{itemize}

An optimal greedy solution to this problem is the select the interval $[s(i), f(i)]$ that has the smallest $f(i)$ (finishes first)

This can be expressed formally as: \newline

\begin{algorithmic}
    \State Initially let R be the set of all request, and let A be empty
    \While{R is not yet empty}
        \State Choose a request $i \in R$ that has the smallest finishing time
        \State Add request i to A
        \State Delete all requests from R that are not compatible with request i
    \EndWhile
    \State Return the set A as the set of accepted requests 
    
\end{algorithmic}


\end{document}
